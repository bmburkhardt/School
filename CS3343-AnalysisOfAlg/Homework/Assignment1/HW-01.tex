\documentclass[12pt]{elsart}
\usepackage{amsmath}
\usepackage{amssymb}
\usepackage{program}
\newcommand{\field}[1]{\mathbb{#1}}

\usepackage{algorithm}
\usepackage{algpseudocode}

%\vspace{10 mm}

\begin{document}

\pagestyle{empty}
Bryan Burkhardt - XMV643\\
CS3343 Section 01\\
02 Sep 2017\\

\begin{center}
\Large  CS3343 Analysis of Algorithms Fall 2017 \\
\large {\bf Homework 1}\\
\normalsize Due 9/8/17 before 11:59pm
\end{center}

Justify all of your answers with comments/text in order to receive full credit.  Completing the assignment in Latex will earn you extra credit on Midterm 1.

{\bf 1. Sorting (8 points)}

Consider the sorting algorithm below which
sorts the array $A[1\ldots n]$ into increasing order by repeatedly bubbling the
minimum element of the remaining array to the left.

\begin{algorithm}
\caption{mysterysort(int $A[1\ldots n]$)}
 \begin{algorithmic}
 %
 \State $i = 1$;
 \While{$i<=n$}
 \State //(I) The elements in $A[1 \ldots (i-1)]$  of the array are in sorted order and are all smaller than the elements in $A[i \ldots n]$
    \State $k = n$;
    \While{$(k >= i+1)$}
	\State //Inner while loop moves the smallest element in $A[i \ldots n]$ to $A[i]$
          \If{$A[k-1]>A[k]$}
              \State swap $A[k]$ with $A[k-1]$
          \EndIf
       \State $k--$;
     \EndWhile
    \State $i++$;
  \EndWhile
\end{algorithmic}
\end{algorithm}

\begin{enumerate}
   \item (2 points) Consider running the above sort on the array $[5,3,1,4,2]$.  Show the sequence of changes which the algorithm makes to the array.\\ 
$[5,3,1,{\bf 4,2}]\rightarrow$ Swap\\
$[5,3,{\bf 1,2},4]\rightarrow$ Continue\\
$[5,{\bf 3,1},2,4]\rightarrow$ Swap\\
$[{\bf 5,1},3,2,4]\rightarrow$ Swap\\
$[1,5,3,2,4]         \rightarrow$ End of first iteration\\
$[1,5,3,{\bf 2,4}]\rightarrow$ Continue\\
$[1,5,{\bf 3,2},4]\rightarrow$ Swap\\
$[1,{\bf 5,2},3,4]\rightarrow$ Swap\\
$[1,2,5,3,4]         \rightarrow$ End of second iteration\\
$[1,2,5,{\bf 3,4}]\rightarrow$ Continue\\
$[1,2,{\bf 5,3},4]\rightarrow$ Swap\\
$[1,2,3,5,4]         \rightarrow$ End of third iteration\\
$[1,2,3,{\bf 5,4}]\rightarrow$ Swap\\
$[1,2,3,4,5]         \rightarrow$ Sorted\\
  \item (4 points) Use induction to prove the loop invariant (I) is true and then use this to prove the correctness of the algorithm.  Specifically complete the following:
\begin{enumerate}
   \item {\bf Base case:}\\\\
	{\bf Solution:}\\
	$i = 1$\\ 
	$A[1\ldots (i-1)] = A[1\ldots 0]$ = [  ]\\ 
	$A[i \ldots n] = [5,3,1,4,2]$\\ 
	Base case holds true since $A[1\ldots (i-1)]$ = [  ] is sorted and\\
	$A[1\ldots (i-1)]$ = [  ] $<$ $A[i \ldots n] = [5,3,1,4,2]$\\
  \item {\bf Inductive step:}\\ 
	({\bf Hint:} you can assume that the inner loop moves the smallest element in $A[i \ldots n]$ to $A[i]$)\\\\
	{\bf Solution:}\\
	{\bf Assume:} The elements in $A[1 \ldots (i_{old}-1)]$  of the array are in sorted order and are all smaller than the elements in $A[i \ldots n]$\\ \\
	{\bf Prove:} The elements in $A[1 \ldots (i_{new}-1)]$  of the array are in sorted order and are all smaller than the elements in $A[i \ldots n]$\\\\
	$i_{new} = i_{old}+1$\\
	$A[1 \ldots (i_{new}-1)]$\\
	$=A[1 \ldots (i_{old}+1-1)]$\\
	$=A[1 \ldots (i_{old})]$\\\\
	Therefore, because the inner loop moves the smallest element in $A[i_{old} \ldots n]$ to $A[i_{old}], A[1 \ldots i_{old}]$ will always be sorted and smaller than elements in $A[i \ldots n].$ Because the loop invariant states the elements in the sorted part of the array are less than the elements in the unsorted part, $i_{new}$ will always be greater than $i_{old}.$\\

\newpage
   \item {\bf Termination step:}\\\\
	{\bf Solution:}\\
	The loop terminates when $i <= n$ is false. Therefore $i > n$ when we fall out of the loop. $i = n + 1$ when we leave the loop. If we plug $i$ into the loop invariant $A[1 \ldots (i-1)]$ which is sorted we get $A[1 \ldots (n+1)-1] = A[1 \ldots n]$ which is sorted by the loop invariant. Thus, the algorithm is correct.\\
\end{enumerate}
   \item (2 points) Give the best-case and worst-case runtimes of this sort in asymptotic (i.e., $O$) notation.
\end{enumerate}

\vspace*{0.5cm}


{\bf 2. Asymptotic Notation (8 points)}\\
Show the following using the definitions of $O$, $\Omega$, and $\Theta$.

\begin{enumerate}
   \item (2 points) $2n^3+n^2+4\in \Theta(n^3)$\\
	{\bf Solution:}
	\begin{itemize}
	\item Show $O: 2n^3 + n^2 +4 <= cn^3$\\
		$2n^3+n^2+4 <= 2n^3+{\bf (n)}n^2+4 \hfill n>=1$\\
		$3n^3+4 <= 3n^3+{\bf (n^3)} \hfill n>= \sqrt[3]{4}$\\
		$4n^3 <= cn^3 \hfill c=4, n>=\sqrt[3]{4}$\\
	\item Show $\Omega: 2n^3+n^2+4 >= cn^3$\\
		$2n^3+n^2+4{\bf (-4)} >= 2n^3+n^2$\\
		$2n^3+n^2{\bf (-n^2)} >= 2n^3$\\
		$2n^3 >= cn^3 \hfill c=2, n>=0$\\\\
	Therefore $2n^3+n^2+4\in \Theta(n^3)$\\
	\end{itemize}
   \item (2 points) $3n^4-9n^2+4n\in \Theta(n^4)$\\  ({\bf Hint:} careful with the negative number)\\
	{\bf Solution:}
	\begin{itemize}
	\item Show $O: 3n^4-9n^2+4n <= cn^4$\\
		$3n^4-9n^2+4n <= 3n^4-9n^2+{\bf (n^3)}n \hfill n>=\sqrt[3]{4}$\\
		$4n^4-9n^2 <= 4n^4-9n^2{\bf (+9n^2)}$\\
		$4n^4 <= cn^4 \hfill c=4, n>=\sqrt[3]{4}$\\\\
	\item Show $\Omega: 3n^4-9n^2+4n >= cn^4$\\
		$3n^4-9n^2+4n >= 3n^4-9n^2+4n{\bf (-4n)}$\\
		$3n^4-9n^2 >= 3n^4-{\bf (n^2)}n^2 \hfill n>=3$\\
		$2n^4 >= cn^4 \hfill c=2, n>=3$\\\\
	Therefore $3n^4-9n^2+4n\in \Theta(n^4)$\\
	\end{itemize}
   \item (4 points) Suppose $f(n)\in O(g_1(n))$ and $f(n)\in O(g_2(n))$.  Which of the following are true?  Justify your answers using the definition of $O$.  Give a counter example if it is false.
\begin{enumerate}
   \item $f(n)\in O( \ 5*g_1(n)+100\ )$\\
	{\bf Solution:}\\
	{\bf Assume:} $f(n)\in O(g_1(n)) \Leftrightarrow f(n)<=c_1g_1(n)$ for all $n >= n_1$\\
	{\bf Prove:} $f(n) \in O(5g_1(n)+100) \Leftrightarrow f(n) <= c_3(5g_1(n)+100)$\\\\
	$\rightarrow LHS = f(n) <= c_1g_1(n)$\\
	$\rightarrow c_1g_1(n) <= c_15g_1(n)$\\
	$\rightarrow c_15g_1(n) <= c_15g_1(n)+100c_1$\\
	$\rightarrow c_15g_1(n)+100c_1 <= c_1(5g_1(n)+100) = RHS$\\
	$\boxed{c_1(5g_1(n)+100)} \hfill \boxed{c_3=c_1, n_3=n_1}$\\\\
	Therefore $f(n)\in O( \ 5*g_1(n)+100\ )$\\
   \item $f(n)\in O( \ g_1(n)+g_2(n)\ )$\\
	{\bf Solution:}\\
	{\bf Assume:} $f(n)\in O(g_1(n)$ and $f(n)\in O(g_2(n)) \Leftrightarrow$\\
	$ f(n) <= c_1g_1(n)$ for all $n>=n_1$ and\\ 
	$f(n) <= c_2g_2(n)$ for all $n>=n_2$\\\\
	{\bf Prove:} $f(n)\in O(g_1(n)+g_2(n))\Leftrightarrow f(n) <= c_3(g_1(n)+g_2(n))$\\\\
	$\rightarrow LHS = f(n) <= c_1g_1(n)$\\
	$\rightarrow c_1g_1(n) <= c_1g_1(n) + c_2g_2(n)$\\
	$\rightarrow c_1g_1(n) + c_2g_2(n) <= c_1g_1(n)c_2 + c_2g_2(n)c_1$\\
	$\rightarrow c_1g_1(n)c_2 + c_2g_2(n)c_1 = c_1c_2(g_1(n)+g_2(n))$\\
	$\rightarrow c_1c_2(g_1(n)+g_2(n)) <= c_3(g_1(n)+g_2(n)) = RHS$\\
	$\rightarrow \boxed{c_3(g_1(n)+g_2(n))}\hfill \boxed{c_3=c_1c_2, n_3=\max(n_1,n_2)}$\\\\
	Therefore $f(n)\in O( \ g_1(n)+g_2(n)\ )$\\
   \item $f(n)\in O( \ \frac{g_1(n)}{g_2(n)}\ )$\\
	{\bf Solution:}\\
	False. Counterexample:\\
	$f(n)=n$\\
	$g_1(n)=n$\\
	$g_2(n)=n^2$\\
	$f(n) \in g_1(n)$ and $f(n) \in g_2(n)$\\
	$\boxed{f(n) \not\in \frac{g_1(n)}{g_2(n)}}$\\\\
   \item $f(n)\in O( \ \max(g_1(n), g_2(n))\ )$\\
	{\bf Solution:}\\
	{\bf Assume:} $f(n)\in O(g_1(n)$ and $f(n)\in O(g_2(n)) \Leftrightarrow$\\
	$ f(n) <= c_1g_1(n)$ for all $n>=n_1$ and\\ 
	$f(n) <= c_2g_2(n)$ for all $n>=n_2$\\\\
	{\bf Prove:} $f(n)\in O( \ \max(g_1(n), g_2(n))\ ) \Leftrightarrow f(n) <= c_3( \ \max(g_1(n), g_2(n))\ )$\\\\
	$\rightarrow LHS = f(n) <= c_1g_1(n)$\\
	$\rightarrow c_1g_1(n) <= \max(c_1g_1(n), c_2g_2(n))$\\
	$\rightarrow \max(c_1g_1(n), c_2g_2(n)) <= \max(c_1g_1(n)c_2, c_2g_2(n)c_1)$\\
	$\rightarrow \max(c_1g_1(n)c_2, c_2g_2(n)c_1) <= c_1c_2*\max(g_1(n), g_2(n))$\\
	$\rightarrow c_1c_2*\max(g_1(n), g_2(n)) <= c_3*\max(g_1(n), g_2(n)) = RHS$\\
	$\rightarrow \boxed{c_3( \ \max(g_1(n), g_2(n))\ )} \hfill \boxed{c_3=c_1c_2, n_3=\max(n_1,n_2)}$\\\\
	Therefore $f(n)\in O( \ \max(g_1(n), g_2(n))\ )$\\
\end{enumerate}
\end{enumerate}

{\bf 3. Summations (4 points)}

Find the order of growth of the following sums.

\begin{enumerate}
\item $\sum\limits_{i=5}^{n} (4i+1). $\\
	{\bf Solution:}\\\\
	$\sum\limits_{i=5}^{n} 4i+1 $\\
	$= \sum\limits_{i=5}^{n} 4i + \sum\limits_{i=5}^{n} 1$\\
	$= 4 \sum\limits_{i=5}^{n} i + (n-4)$\\
	$= 4\frac{(n+5)(n-4)}{2} + (n-4)$\\
	$= 2(n^2+n-20) + (n-4)$\\
	$= 2n^2+3n-44$\\
	Therefore $\sum\limits_{i=5}^{n} (4i+1)\in \Theta(n^2)$\\\\

\newpage

\item $\sum\limits_{i=0}^{ \log_2 (n) } 2^i$  (for simplicity you can assume $n$ is a power of 2)
	\\{\bf Solution:}\\\\
	$\sum\limits_{i=0}^{ \log_2 (n) } 2^i = 2^0+2^1+2^2+\ldots+2^{\log_2(n-1)}+2^{\log_2(n)}$\\
	$= 1+2+4+8+\ldots+\frac{n}{2}+n$\\
	$= 2n-1$\\
	Therefore $\sum\limits_{i=0}^{ \log_2 (n) } 2^i \in \Theta(n)$\\
\end{enumerate}


\end{document}